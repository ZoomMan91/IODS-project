% Options for packages loaded elsewhere
\PassOptionsToPackage{unicode}{hyperref}
\PassOptionsToPackage{hyphens}{url}
%
\documentclass[
]{article}
\usepackage{lmodern}
\usepackage{amssymb,amsmath}
\usepackage{ifxetex,ifluatex}
\ifnum 0\ifxetex 1\fi\ifluatex 1\fi=0 % if pdftex
  \usepackage[T1]{fontenc}
  \usepackage[utf8]{inputenc}
  \usepackage{textcomp} % provide euro and other symbols
\else % if luatex or xetex
  \usepackage{unicode-math}
  \defaultfontfeatures{Scale=MatchLowercase}
  \defaultfontfeatures[\rmfamily]{Ligatures=TeX,Scale=1}
\fi
% Use upquote if available, for straight quotes in verbatim environments
\IfFileExists{upquote.sty}{\usepackage{upquote}}{}
\IfFileExists{microtype.sty}{% use microtype if available
  \usepackage[]{microtype}
  \UseMicrotypeSet[protrusion]{basicmath} % disable protrusion for tt fonts
}{}
\makeatletter
\@ifundefined{KOMAClassName}{% if non-KOMA class
  \IfFileExists{parskip.sty}{%
    \usepackage{parskip}
  }{% else
    \setlength{\parindent}{0pt}
    \setlength{\parskip}{6pt plus 2pt minus 1pt}}
}{% if KOMA class
  \KOMAoptions{parskip=half}}
\makeatother
\usepackage{xcolor}
\IfFileExists{xurl.sty}{\usepackage{xurl}}{} % add URL line breaks if available
\IfFileExists{bookmark.sty}{\usepackage{bookmark}}{\usepackage{hyperref}}
\hypersetup{
  hidelinks,
  pdfcreator={LaTeX via pandoc}}
\urlstyle{same} % disable monospaced font for URLs
\usepackage[margin=1in]{geometry}
\usepackage{color}
\usepackage{fancyvrb}
\newcommand{\VerbBar}{|}
\newcommand{\VERB}{\Verb[commandchars=\\\{\}]}
\DefineVerbatimEnvironment{Highlighting}{Verbatim}{commandchars=\\\{\}}
% Add ',fontsize=\small' for more characters per line
\usepackage{framed}
\definecolor{shadecolor}{RGB}{248,248,248}
\newenvironment{Shaded}{\begin{snugshade}}{\end{snugshade}}
\newcommand{\AlertTok}[1]{\textcolor[rgb]{0.94,0.16,0.16}{#1}}
\newcommand{\AnnotationTok}[1]{\textcolor[rgb]{0.56,0.35,0.01}{\textbf{\textit{#1}}}}
\newcommand{\AttributeTok}[1]{\textcolor[rgb]{0.77,0.63,0.00}{#1}}
\newcommand{\BaseNTok}[1]{\textcolor[rgb]{0.00,0.00,0.81}{#1}}
\newcommand{\BuiltInTok}[1]{#1}
\newcommand{\CharTok}[1]{\textcolor[rgb]{0.31,0.60,0.02}{#1}}
\newcommand{\CommentTok}[1]{\textcolor[rgb]{0.56,0.35,0.01}{\textit{#1}}}
\newcommand{\CommentVarTok}[1]{\textcolor[rgb]{0.56,0.35,0.01}{\textbf{\textit{#1}}}}
\newcommand{\ConstantTok}[1]{\textcolor[rgb]{0.00,0.00,0.00}{#1}}
\newcommand{\ControlFlowTok}[1]{\textcolor[rgb]{0.13,0.29,0.53}{\textbf{#1}}}
\newcommand{\DataTypeTok}[1]{\textcolor[rgb]{0.13,0.29,0.53}{#1}}
\newcommand{\DecValTok}[1]{\textcolor[rgb]{0.00,0.00,0.81}{#1}}
\newcommand{\DocumentationTok}[1]{\textcolor[rgb]{0.56,0.35,0.01}{\textbf{\textit{#1}}}}
\newcommand{\ErrorTok}[1]{\textcolor[rgb]{0.64,0.00,0.00}{\textbf{#1}}}
\newcommand{\ExtensionTok}[1]{#1}
\newcommand{\FloatTok}[1]{\textcolor[rgb]{0.00,0.00,0.81}{#1}}
\newcommand{\FunctionTok}[1]{\textcolor[rgb]{0.00,0.00,0.00}{#1}}
\newcommand{\ImportTok}[1]{#1}
\newcommand{\InformationTok}[1]{\textcolor[rgb]{0.56,0.35,0.01}{\textbf{\textit{#1}}}}
\newcommand{\KeywordTok}[1]{\textcolor[rgb]{0.13,0.29,0.53}{\textbf{#1}}}
\newcommand{\NormalTok}[1]{#1}
\newcommand{\OperatorTok}[1]{\textcolor[rgb]{0.81,0.36,0.00}{\textbf{#1}}}
\newcommand{\OtherTok}[1]{\textcolor[rgb]{0.56,0.35,0.01}{#1}}
\newcommand{\PreprocessorTok}[1]{\textcolor[rgb]{0.56,0.35,0.01}{\textit{#1}}}
\newcommand{\RegionMarkerTok}[1]{#1}
\newcommand{\SpecialCharTok}[1]{\textcolor[rgb]{0.00,0.00,0.00}{#1}}
\newcommand{\SpecialStringTok}[1]{\textcolor[rgb]{0.31,0.60,0.02}{#1}}
\newcommand{\StringTok}[1]{\textcolor[rgb]{0.31,0.60,0.02}{#1}}
\newcommand{\VariableTok}[1]{\textcolor[rgb]{0.00,0.00,0.00}{#1}}
\newcommand{\VerbatimStringTok}[1]{\textcolor[rgb]{0.31,0.60,0.02}{#1}}
\newcommand{\WarningTok}[1]{\textcolor[rgb]{0.56,0.35,0.01}{\textbf{\textit{#1}}}}
\usepackage{graphicx,grffile}
\makeatletter
\def\maxwidth{\ifdim\Gin@nat@width>\linewidth\linewidth\else\Gin@nat@width\fi}
\def\maxheight{\ifdim\Gin@nat@height>\textheight\textheight\else\Gin@nat@height\fi}
\makeatother
% Scale images if necessary, so that they will not overflow the page
% margins by default, and it is still possible to overwrite the defaults
% using explicit options in \includegraphics[width, height, ...]{}
\setkeys{Gin}{width=\maxwidth,height=\maxheight,keepaspectratio}
% Set default figure placement to htbp
\makeatletter
\def\fps@figure{htbp}
\makeatother
\setlength{\emergencystretch}{3em} % prevent overfull lines
\providecommand{\tightlist}{%
  \setlength{\itemsep}{0pt}\setlength{\parskip}{0pt}}
\setcounter{secnumdepth}{-\maxdimen} % remove section numbering

\author{}
\date{\vspace{-2.5em}}

\begin{document}

\hypertarget{insert-chapter-2-title-here}{%
\section{Insert chapter 2 title
here}\label{insert-chapter-2-title-here}}

\emph{Describe the work you have done this week and summarize your
learning.}

\begin{itemize}
\tightlist
\item
  Describe your work and results clearly.
\item
  Assume the reader has an introductory course level understanding of
  writing and reading R code as well as statistical methods.
\item
  Assume the reader has no previous knowledge of your data or the more
  advanced methods you are using.
\end{itemize}

Today is:

\begin{Shaded}
\begin{Highlighting}[]
\KeywordTok{date}\NormalTok{()}
\end{Highlighting}
\end{Shaded}

\begin{verbatim}
## [1] "Tue Nov  3 23:34:40 2020"
\end{verbatim}

\hypertarget{task-1.-read-and-explore-the-data.}{%
\paragraph{Task 1. Read and explore the
data.}\label{task-1.-read-and-explore-the-data.}}

Reading data from local directory.

\begin{Shaded}
\begin{Highlighting}[]
\NormalTok{students2014<-}\KeywordTok{read.table}\NormalTok{(}\DataTypeTok{file=}\StringTok{"data/learning2014.txt"}\NormalTok{,}\DataTypeTok{header=}\NormalTok{T, }\DataTypeTok{sep=}\StringTok{"}\CharTok{\textbackslash{}t}\StringTok{"}\NormalTok{)}
\end{Highlighting}
\end{Shaded}

Success! Let's check the data structure using \textbf{R} functions
\textbf{dim} and \textbf{str}.

\begin{Shaded}
\begin{Highlighting}[]
\KeywordTok{dim}\NormalTok{(students2014)}
\end{Highlighting}
\end{Shaded}

\begin{verbatim}
## [1] 166   7
\end{verbatim}

\begin{Shaded}
\begin{Highlighting}[]
\KeywordTok{str}\NormalTok{(students2014)}
\end{Highlighting}
\end{Shaded}

\begin{verbatim}
## 'data.frame':    166 obs. of  7 variables:
##  $ gender  : chr  "F" "M" "F" "M" ...
##  $ Age     : int  53 55 49 53 49 38 50 37 37 42 ...
##  $ Attitude: int  37 31 25 35 37 38 35 29 38 21 ...
##  $ deep    : num  3.58 2.92 3.5 3.5 3.67 ...
##  $ stra    : num  3.38 2.75 3.62 3.12 3.62 ...
##  $ surf    : num  2.58 3.17 2.25 2.25 2.83 ...
##  $ Points  : int  25 12 24 10 22 21 21 31 24 26 ...
\end{verbatim}

The data has 7 variables from 166 students. Variables \textbf{deep},
\textbf{strategic} and \textbf{surface} learning were created based on
survey done in 2014. Method to calculate variables is described
\href{https://www.mv.helsinki.fi/home/kvehkala/JYTmooc/JYTOPKYS2-meta.txt}{here}.
Rest variables are \textbf{gender}, \textbf{Age}, global
\textbf{Attitude} towards statistics, and final exam \textbf{Points}.

\hypertarget{task-2.-graphcal-overview-of-the-data.}{%
\paragraph{Task 2. Graphcal overview of the
data.}\label{task-2.-graphcal-overview-of-the-data.}}

Inbuild \textbf{R} function \textbf{plot} quite ugly. Therefore I would
use \href{https://ggplot2.tidyverse.org/reference/ggplot.html}{ggplot2}
as much elegant way to overview the data.

Let's access ggplot package first.

\begin{Shaded}
\begin{Highlighting}[]
\KeywordTok{library}\NormalTok{(ggplot2)}
\end{Highlighting}
\end{Shaded}

\href{https://ggobi.github.io/ggally/}{GGally} package will be useful
also.

\begin{Shaded}
\begin{Highlighting}[]
\KeywordTok{library}\NormalTok{(GGally)}
\end{Highlighting}
\end{Shaded}

\begin{verbatim}
## Registered S3 method overwritten by 'GGally':
##   method from   
##   +.gg   ggplot2
\end{verbatim}

Now let's create pairwise plot presenting relationships among variables.

\begin{Shaded}
\begin{Highlighting}[]
\KeywordTok{ggpairs}\NormalTok{(students2014, }\DataTypeTok{mapping =} \KeywordTok{aes}\NormalTok{(}\DataTypeTok{col=}\NormalTok{gender,}\DataTypeTok{alpha=}\FloatTok{0.3}\NormalTok{), }\DataTypeTok{lower =} \KeywordTok{list}\NormalTok{(}\DataTypeTok{combo =} \KeywordTok{wrap}\NormalTok{(}\StringTok{"facethist"}\NormalTok{, }\DataTypeTok{bins =} \DecValTok{20}\NormalTok{)))}
\end{Highlighting}
\end{Shaded}

\includegraphics{chapter2_files/figure-latex/unnamed-chunk-6-1.pdf}

Here we go again\ldots{}

\end{document}
